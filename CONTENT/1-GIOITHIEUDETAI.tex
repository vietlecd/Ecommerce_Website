\section{Giới thiệu đề tài}
\subsection{Tổng quan đề tài}
Trong bối cảnh chuyển đổi số mạnh mẽ diễn ra trên toàn cầu, thương mại điện tử (TMĐT) đã và đang trở thành xu hướng tất yếu trong hoạt động kinh doanh, đặc biệt tại Việt Nam – thị trường có quy mô TMĐT đạt 16,4 tỷ USD vào năm 2022 và dự kiến tăng trưởng khoảng 20\%\ vào năm 2023. Trong đó, mô hình B2C (Business-to-Consumer) ngày càng khẳng định vai trò quan trọng nhờ khả năng kết nối trực tiếp giữa doanh nghiệp và người tiêu dùng, giúp giảm thiểu chi phí trung gian, đồng thời mở rộng khả năng tiếp cận khách hàng không bị giới hạn bởi không gian và thời gian. Lĩnh vực thời trang giày dép là một trong những phân khúc đầy tiềm năng khi Việt Nam sở hữu dân số trẻ với hơn 60\%\ người dưới 35 tuổi (theo Tổng cục Thống kê, 2023), kéo theo nhu cầu lớn về các sản phẩm giày dép thời trang, đa dạng mẫu mã và chất lượng. Tuy nhiên, người tiêu dùng vẫn gặp nhiều khó khăn trong việc tìm kiếm sản phẩm chính hãng, so sánh giá cả hay cập nhật các xu hướng mới nhất.\\

Xuất phát từ thực tiễn này, website B2C bán giày được xây dựng nhằm giải quyết các vấn đề trên bằng cách cung cấp một nền tảng mua sắm trực tuyến tiện lợi, minh bạch và chuyên nghiệp. Website không chỉ giúp khách hàng dễ dàng tìm kiếm, lọc sản phẩm theo nhiều tiêu chí khác nhau, mà còn hỗ trợ đặt hàng nhanh chóng, theo dõi đơn hàng theo thời gian thực, cũng như nhận được các gợi ý phù hợp dựa trên hành vi mua sắm cá nhân. Về phía doanh nghiệp, hệ thống giúp tối ưu hóa quản lý kho hàng, chăm sóc khách hàng và phân tích dữ liệu để đưa ra chiến lược kinh doanh hiệu quả. Dự án tập trung phát triển website với các tính năng chính như: hệ thống quản lý sản phẩm đa dạng, tích hợp thanh toán trực tuyến an toàn, hệ thống quản lý đơn hàng thông minh, giao diện người dùng thân thiện và responsive, hệ thống quản trị nội dung linh hoạt, cùng các công cụ marketing và phân tích dữ liệu. Việc triển khai dự án này không chỉ đáp ứng nhu cầu thiết thực của thị trường mà còn góp phần thúc đẩy quá trình chuyển đổi số trong lĩnh vực thương mại điện tử tại Việt Nam, mở ra cơ hội phát triển bền vững cho các doanh nghiệp trong ngành thời trang giày dép.\\
\subsection{Sự cần thiết đề tài}
\begin{enumerate}
    \item \textit{Xu hướng chuyển đổi số trong thương mại:}
    Thị trường giày dép Việt Nam tăng trưởng 15-20\% mỗi năm, kéo theo nhu cầu mua sắm trực tuyến tăng mạnh.
    Theo Nielsen (2023), 75\% người tiêu dùng Việt Nam ưu tiên mua sắm online, đặc biệt là giới trẻ - chiếm 70\% dân số.
    
    \item \textit{Thực trạng thị trường giày dép:}
    Hệ thống cửa hàng truyền thống phân tán, thiếu minh bạch về chất lượng và giá cả.
    Người tiêu dùng gặp khó khăn khi so sánh sản phẩm giữa các cửa hàng và tìm kiếm thông tin chính xác.
    Chi phí mặt bằng và vận hành cửa hàng vật lý cao làm hạn chế đa dạng hóa sản phẩm.
\end{enumerate}
\subsection{Lợi ích và tiềm năng phát triển}
\begin{enumerate}
    \item \textit{Đối với doanh nghiệp:}
    Giảm chi phí vận hành 35-45\% so với mô hình truyền thống.
    Mở rộng thị trường toàn quốc và tiềm năng vươn ra quốc tế.
    Thu thập và phân tích dữ liệu khách hàng theo thời gian thực.
    
    \item \textit{Đối với người tiêu dùng:}
    Tiết kiệm 50-70\% thời gian mua sắm so với phương thức truyền thống.
    Dễ dàng so sánh giữa 1000+ mẫu giày từ nhiều thương hiệu.
    Đảm bảo chất lượng chính hãng với chính sách bảo hành rõ ràng.
    
    \item \textit{Tiềm năng phát triển:}
    Thị trường giày dép Việt Nam dự kiến đạt 3.5 tỷ USD vào 2025 (Statista).
    Tỷ lệ người dùng smartphone đạt 82\% dân số, tạo điều kiện mua sắm di động.
    90\% khách hàng trẻ tuổi ưa chuộng thanh toán qua ví điện tử hoặc thẻ quốc tế.
\end{enumerate}

Tóm lại, trong bối cảnh chuyển đổi số 4.0, việc phát triển website bán giày trực tuyến không chỉ đáp ứng xu hướng thị trường mà còn là giải pháp chiến lược giúp doanh nghiệp tối ưu hóa hoạt động kinh doanh và nâng cao trải nghiệm khách hàng trong ngành thời trang - một trong những lĩnh vực có tốc độ tăng trưởng nhanh nhất hiện nay.