\section{Cơ sở lý thuyết}
\begin{figure}[h]
    \centering
    \includegraphics[width=\textwidth]{IMG/Html-Css-Js.jpg} 
    \caption{HTML-CSS-JS.}
    \label{fig:my_label}
\end{figure}
\subsection{HTML}
HTML là viết tắt của cụm từ HyperText Markup Language, được sử dụng để tạo nên trang web. Trên một trang web có thể chứa nhiều trang và mỗi trang là một tài liệu HTML. Cha đẻ của HTML là Tim Berners-Lee, cũng chính là người sinh ra World Wide Web và là chủ tịch của World Wide Web Consortium (W3C - tổ chức thiết lập ra các chuẩn trên môi trường internet).\\

HTML là ngôn ngữ đánh dấu siêu văn bản, nó có vai trò xây dựng cấu trúc siêu văn bản trên một website, hoặc khai báo các tập tin kỹ thuật số như video, hình ảnh, nhạc. Nhóm chúng em đã tìm hiểu HTML (tìm hiểu cách chạy một file HTML, các thẻ HTML trên trang chủ W3C) để ứng dụng vào việc xây dựng cấu trúc cho các trang web chính của một khách sạn như: trang chủ, trang shop, trang thông tin khách hàng, trang giỏ hàng,... Phần lớn HTML đóng vai trò đưa ra cấu trúc cho trang web trong đồ án và xây dựng nên các trang web tĩnh. Cùng với CSS và JavaScript, HTML tạo ra bộ ba nền tảng kỹ thuật cho các website.\\

Với HTML ta có thể:
\begin{itemize}
    \item Thêm tiêu đề, định dạng đoạn văn, ngắt dòng điều khiển.
    \item Tạo danh sách, nhấn mạnh văn bản, tạo ký tự đặc biệt, chèn hình ảnh, tạo liên kết.
    \item Xây dựng bảng, điều khiển một số kiểu mẫu.
\end{itemize} 

Nhìn chung, HTML là ngôn ngữ markup, dễ học, dễ hiểu, dễ áp dụng. Tuy nhiên, một website được viết bằng HTML rất đơn giản. Để gây hứng thú với người truy cập, website cần có sự hỗ trợ của CSS và JavaScript.\\

\subsection{CSS}
CSS là viết tắt của Cascading Style Sheets, nó là ngôn ngữ được sử dụng để tìm và định dạng lại các phần tử được tạo bởi HTML. Nói ngắn gọn hơn là ngôn ngữ tạo phong cách cho trang web. CSS được phát triển bởi W3C vào năm 1996. Với CSS chúng ta có thể:
\begin{itemize}
    \item Tạo phong cách và định kiểu cho những yếu tố được viết dưới dạng ngôn ngữ đánh dấu, như HTML.
    \item Nội dung trang web sẽ tách bạch hơn trong việc định dạng hiển thị, từ đó quá trình cập nhật sẽ dễ dàng hơn.
    \item Tiết kiệm công sức nhờ điều khiển định dạng của nhiều trang web.
    \item Phân biệt cách hiển thị của trang web với nội dung chính của trang bằng cách điều khiển bố cục, màu sắc và font chữ.
\end{itemize}

Có thể nói, CSS gần như tạo nên bộ mặt của một website.\\

\subsection{JavaScript}
JS (tên đầy đủ là JavaScript) có tác dụng giúp chuyển website từ trạng thái tĩnh sang động, tạo tương tác để cải thiện hiệu suất máy chủ và nâng cao trải nghiệm người dùng. Hiểu đơn giản, JavaScript là ngôn ngữ được sử dụng rộng rãi khi kết hợp với HTML/CSS để thiết kế web động. Bên cạnh HTML và CSS, JavaScript cũng là một ngôn ngữ lập trình web phổ biến, được sử dụng rộng rãi trong suốt 20 năm qua. Tính đến 2016, có đến 92\% trang web hiện nay đang sử dụng JavaScript. Sử dụng JavaScript, ta sẽ:
\begin{itemize}
    \item Dễ dàng bắt đầu với các bước nhỏ, với thư viện ảnh, bố cục có tính thay đổi ... nhờ sự linh hoạt của JavaScript.
    \item Tăng cường các hành vi và kiểm soát mặc định của trình duyệt.
    \item Thông qua JavaScript, ta có thể kiểm tra dữ liệu đầu vào, nhằm giảm bớt công việc kiểm tra thủ công như kiểm tra tính hợp lệ của thông tin khách hàng, kiểm tra thông tin sản phẩm.
    \item JavaScript khá linh hoạt và có thể sử dụng ở nhiều nền tảng, trình duyệt và có thể được biên dịch bởi HTML trình duyệt web. Ta có thể truy cập và tương tác với website hiệu quả hơn, nhờ đặc tính gọn nhẹ mà chúng sẽ cho phép thực hiện các tác vụ trên trang web nhanh hơn.
\end{itemize}

Tóm lại: Tạo “sườn” web bằng HTML, làm cho trang web có nhiều màu sắc hơn bằng CSS, và tạo tính năng “động” cho trang web bằng JavaScript.

\subsection{PHP}
\begin{figure}[h]
    \centering
    \includegraphics[width=\textwidth]{IMG/php-mysql.jpg} 
    \caption{PHP và MySQL.}
    \label{fig:my_label}
\end{figure}

Với một ngôn ngữ làm web phổ biến như PHP, việc học tập nó là rất dễ dàng vì nguồn tài liệu tham khảo rất dồi dào trên Internet. Bản thân PHP có cú pháp và cấu trúc đơn giản, có thể nhanh chóng làm quen và sau khi làm quen, PHP có thể học thêm Framework Laravel.\\
\subsection*{Ưu điểm}
\begin{itemize}
    \item \textbf{Dễ học và sử dụng:} PHP là ngôn ngữ lập trình phổ biến với cú pháp đơn giản, phù hợp cho người mới bắt đầu.
    \item \textbf{Hiệu năng cao:} Được tối ưu cho phát triển web, PHP hoạt động nhanh và hiệu quả trên các ứng dụng server-side.
    \item \textbf{Hỗ trợ cộng đồng rộng lớn:} PHP có cộng đồng phát triển mạnh mẽ, cung cấp nhiều tài liệu, hướng dẫn và thư viện hỗ trợ.
    \item \textbf{Tương thích đa nền tảng:} PHP có thể chạy trên hầu hết các hệ điều hành, như Windows, Linux, và macOS.
    \item \textbf{Tích hợp dễ dàng với cơ sở dữ liệu:} PHP hỗ trợ tích hợp với các cơ sở dữ liệu phổ biến như MySQL, PostgreSQL, SQLite.
    \item \textbf{Miễn phí:} PHP là mã nguồn mở và không yêu cầu chi phí bản quyền.
\end{itemize}

\subsection*{Nhược điểm}
\begin{itemize}
    \item Còn hạn chế về cấu trúc của ngữ pháp. Nó không được thiết kế gọn gàng và không được đẹp mắt như những ngôn ngữ lập trình khác.
    \item PHP chỉ có thể hoạt động và sử dụng được trên các ứng dụng trong web. Đó chính là lý do khiến cho ngôn ngữ này khó có thể cạnh tranh được với những ngôn ngữ lập trình khác nếu như muốn phát triển và mở rộng hơn nữa trong lập trình.
\end{itemize}

\subsection{MySQL}
Về MySQL, là một trong những hệ thống quản trị cơ sở dữ liệu quan hệ SQL phổ biến nhất, phối hợp tốt với PHP. Nó được bảo trì bởi Oracle và được cập nhật thường xuyên. MySQL còn hỗ trợ indexing, bảo mật và cấp quyền ở một mức độ đơn giản.
\subsection*{Ưu điểm}
\begin{itemize}
    \item Về Giới hạn: Theo thiết kế, MySQL không có ý định làm tất cả và nó đi kèm với các hạn chế về chức năng mà một vài ứng dụng có thể cần.
    \item Về mức Độ tin cậy: Cách các chức năng cụ thể được xử lý với MySQL (ví dụ tài liệu tham khảo, các giao dịch, kiểm toán, ...) làm cho nó kém tin cậy hơn so với một số hệ quản trị cơ sở dữ liệu quan hệ khác.
    \item Ngoài ra, nếu số bản ghi của bạn lớn dần lên thì việc truy xuất dữ liệu của bạn là khá khó khăn. Khi đó, chúng ta sẽ phải áp dụng nhiều biện pháp để tăng tốc độ truy xuất dữ liệu như là chia tải database này ra nhiều server, hoặc tạo cache MySQL.
\end{itemize}

\subsection*{Nhược điểm của MySQL}
\begin{itemize}
    \item \textbf{Hạn chế về hiệu năng với dữ liệu lớn:} MySQL có thể không hoạt động hiệu quả khi xử lý cơ sở dữ liệu rất lớn hoặc ứng dụng có tải cao, so với các hệ quản trị như PostgreSQL hoặc Oracle.
    \item \textbf{Tính năng hạn chế:} Một số tính năng nâng cao, như hỗ trợ toàn diện cho các truy vấn phức tạp (CTE, window functions) hoặc giao dịch phân tán, chỉ mới được hỗ trợ trong các phiên bản gần đây hoặc còn hạn chế.
    \item \textbf{Thiếu sự tuân thủ hoàn toàn chuẩn SQL:} MySQL không tuân thủ chặt chẽ chuẩn SQL, có thể gây khó khăn khi di chuyển ứng dụng sang các hệ quản trị cơ sở dữ liệu khác.
    \item \textbf{Quản lý giao dịch hạn chế:} Mặc dù InnoDB hỗ trợ giao dịch, các công cụ lưu trữ khác trong MySQL, như MyISAM, không hỗ trợ giao dịch, gây khó khăn cho việc sử dụng nhất quán.
    \item \textbf{Tùy chọn sao lưu còn hạn chế:} Các phương pháp sao lưu mặc định của MySQL không mạnh mẽ bằng một số hệ thống khác, như Oracle hoặc PostgreSQL.
    \item \textbf{Không mạnh mẽ với phân tích dữ liệu:} MySQL không phải là lựa chọn tối ưu cho các ứng dụng phân tích dữ liệu phức tạp, vì thiếu các tính năng tối ưu cho mục đích này.
    \item \textbf{Cần cấu hình thủ công:} Hiệu suất của MySQL phụ thuộc nhiều vào việc cấu hình thủ công, điều này đòi hỏi kiến thức chuyên sâu từ quản trị viên.
    \item \textbf{Không tối ưu cho xử lý phân tán:} MySQL không được thiết kế để hỗ trợ tốt cho các hệ thống cơ sở dữ liệu phân tán, cần các công cụ bổ sung như Galera Cluster.
\end{itemize}

\subsection{Bootstrap}
\begin{center}
\includegraphics[width=\textwidth]{IMG/boostrap.png}\\
\small Bootstrap.
\end{center}
\textbf{Bootstrap} là một framework front-end mã nguồn mở được phát triển bởi Twitter. Nó được thiết kế để hỗ trợ xây dựng các giao diện web hiện đại và tối ưu hóa cho các thiết bị di động. Được giới thiệu lần đầu vào năm 2011, Bootstrap đã trở thành một trong những công cụ phổ biến nhất trong lĩnh vực thiết kế web nhờ tính tiện dụng và linh hoạt.

\textbf{Bootstrap} là một \textit{CSS framework} kết hợp HTML, CSS và JavaScript để cung cấp một bộ công cụ mạnh mẽ giúp lập trình viên và nhà thiết kế tạo giao diện người dùng (UI) nhanh chóng. Bootstrap hoạt động dựa trên triết lý thiết kế \textit{"mobile-first"}, nghĩa là ưu tiên tối ưu hóa cho thiết bị di động trước, sau đó mới mở rộng sang các màn hình lớn hơn.\\

\subsubsection*{Công dụng của Bootstrap}
Bootstrap cung cấp các thành phần và tính năng đa dạng, giúp phát triển giao diện web một cách nhanh chóng và dễ dàng. Một số công dụng chính gồm:

\begin{itemize}
    \item \textbf{Thiết kế giao diện đáp ứng (responsive):} Với hệ thống lưới (\textit{grid system}), Bootstrap cho phép tạo các giao diện hiển thị tốt trên mọi kích thước màn hình.
    \item \textbf{Thư viện thành phần UI phong phú:} Bao gồm các nút bấm, bảng biểu, form, menu, thanh điều hướng (navbar), modal, carousel, và nhiều thành phần khác.
    \item \textbf{Tích hợp sẵn các hiệu ứng JavaScript:} Bootstrap đi kèm với các plugin JavaScript như \textit{dropdowns}, \textit{modals}, \textit{tooltips}, \textit{popovers}, giúp giao diện trở nên sống động hơn.
    \item \textbf{Tùy chỉnh dễ dàng:} Người dùng có thể điều chỉnh các thành phần theo nhu cầu bằng cách ghi đè CSS hoặc sử dụng các biến Sass.
\end{itemize}

\subsubsection*{Ưu điểm của Bootstrap}
\begin{itemize}
    \item \textbf{Dễ sử dụng:} Không yêu cầu kiến thức chuyên sâu về CSS, bạn có thể dễ dàng tạo giao diện đẹp mắt chỉ bằng cách sử dụng các \textit{class} sẵn có.
    \item \textbf{Tiết kiệm thời gian:} Với các thành phần được xây dựng sẵn, Bootstrap giúp giảm đáng kể thời gian phát triển giao diện web.
    \item \textbf{Tính nhất quán:} Bootstrap đảm bảo rằng giao diện của bạn hiển thị nhất quán trên các trình duyệt khác nhau nhờ được kiểm thử rộng rãi.
    \item \textbf{Cộng đồng lớn và tài liệu đầy đủ:} Với tài liệu chi tiết và cộng đồng người dùng rộng lớn, bạn có thể dễ dàng tìm kiếm hướng dẫn, ví dụ và giải pháp khi gặp vấn đề.
    \item \textbf{Tương thích đa nền tảng:} Giao diện được xây dựng bằng Bootstrap tương thích tốt với mọi trình duyệt và thiết bị.
\end{itemize}

\subsubsection*{Hạn chế của Bootstrap}
\begin{itemize}
    \item \textbf{Giao diện tương đồng:} Các trang web sử dụng Bootstrap mặc định có thể dễ bị giống nhau nếu không tùy chỉnh sâu.
    \item \textbf{Hiệu suất:} Việc sử dụng nhiều \textit{class} và thành phần có thể làm tăng kích thước tệp CSS, dẫn đến hiệu suất tải trang bị ảnh hưởng.
    \item \textbf{Đường cong học tập:} Đối với người mới, việc hiểu và sử dụng đầy đủ các tính năng của Bootstrap có thể cần thời gian.
\end{itemize}

\textbf{Kết luận:} Bootstrap là một công cụ mạnh mẽ dành cho việc phát triển giao diện web hiện đại, đặc biệt là khi bạn muốn tối ưu hóa thời gian và chi phí. Với tính linh hoạt và khả năng tùy chỉnh, nó phù hợp cho cả những người mới bắt đầu lẫn các chuyên gia lập trình. Tuy nhiên, việc sử dụng hợp lý và tùy chỉnh theo nhu cầu riêng sẽ giúp bạn tận dụng tối đa các lợi ích mà Bootstrap mang lại.