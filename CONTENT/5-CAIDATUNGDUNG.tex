\section{Hướng dẫn triển khai ứng dụng}

\subsection{Yêu cầu hệ thống}
Trước khi bắt đầu triển khai, đảm bảo hệ thống đã cài đặt đầy đủ các công cụ sau:

\begin{itemize}
  \item Docker Desktop (phiên bản 20.10 trở lên)
  \item Docker Compose (phiên bản 2.0 trở lên)
  \item Git (để clone source code từ repository)
\end{itemize}

\subsection{Quy trình triển khai}

\subsubsection{Bước 1: Cài đặt Docker Desktop}

Tải và cài đặt Docker Desktop phù hợp với hệ điều hành của bạn tại đường dẫn: 

\texttt{https://www.docker.com/products/docker-desktop}

Docker Desktop là một nền tảng tích hợp bao gồm Docker Engine, Docker CLI, Docker 

Compose và Kubernetes, giúp đơn giản hóa quá trình quản lý và vận hành containers trong 

môi trường development.

\subsubsection{Bước 2: Clone repository từ GitHub}

Mở terminal (Linux/macOS) hoặc Command Prompt/PowerShell (Windows) và thực hiện 

lệnh sau:

\begin{verbatim}
  git clone https://github.com/vietlecd/Ecommerce_Website
  cd Ecommerce_Website
\end{verbatim}

\begin{figure}[h!]
  \centering
  \includegraphics[width=0.9\textwidth]{IMG/showimages/git.png} 
  \caption{Repository GitHub chứa source code của dự án}
  \label{fig:github_repo}
\end{figure}

\subsubsection{Bước 3: Cấu hình và khởi động containers}

Di chuyển vào thư mục gốc của project và khởi động các services bằng Docker Compose:

\begin{verbatim}
  docker-compose up -d
\end{verbatim}

Tham số \texttt{-d} (detached mode) sẽ chạy containers ở background. Lệnh này thực hiện các tác 

vụ sau:

\begin{itemize}
  \item Build và khởi động containers cho Web Server (Apache/Nginx) và Database (MySQL)
  \item Cấu hình network cho phép các services giao tiếp với nhau
  \item Mount volumes để đồng bộ source code giữa host và container
  \item Expose ports để truy cập ứng dụng từ host machine
\end{itemize}

\subsubsection{Bước 4: Kiểm tra trạng thái deployment}

Xác nhận tất cả các containers đang chạy ổn định:

\begin{verbatim}
  docker-compose ps
\end{verbatim}

Output mong đợi sẽ hiển thị các services (\texttt{web}, \texttt{db}) với status là \texttt{Up} và port mappings tương 

ứng.

\subsubsection{Bước 5: Truy cập ứng dụng}

Mở trình duyệt web và truy cập địa chỉ sau để sử dụng ứng dụng:

\texttt{http://localhost:8080}

\subsection{Các lệnh quản lý thường dùng}

\begin{itemize}
  \item \textbf{Dừng toàn bộ services:} \texttt{docker-compose down}
  \item \textbf{Xem logs realtime:} \texttt{docker-compose logs -f [service\_name]}
  \item \textbf{Restart services:} \texttt{docker-compose restart [service\_name]}
  \item \textbf{Rebuild containers:} \texttt{docker-compose up -d --build}
  \item \textbf{Import database:} \texttt{docker-compose exec db mysql -u root -p shoesstore < database.sql}
  \item \textbf{Xem resource usage:} \texttt{docker stats}
\end{itemize}
