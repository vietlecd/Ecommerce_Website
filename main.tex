\documentclass[a4paper]{article}
\usepackage{a4wide,amssymb,epsfig,latexsym,multicol,array,hhline,fancyhdr}
\usepackage{vntex}
\usepackage{amsmath}
\usepackage{lastpage}
\usepackage[lined,boxed,commentsnumbered]{algorithm2e}
\usepackage{enumerate}
\usepackage{color}
\usepackage{graphicx}							% Standard graphics package
\usepackage{array}
\usepackage{tabularx, caption}
\usepackage{multirow}
\usepackage{multicol}
\usepackage{rotating}
\usepackage{graphics}
\usepackage{geometry}
\usepackage{setspace}
\usepackage{epsfig}
\usepackage{tikz}
\usetikzlibrary{arrows,snakes,backgrounds}
\usepackage{hyperref}
\hypersetup{urlcolor=blue,linkcolor=black,citecolor=black,colorlinks=true} 
%\usepackage{pstcol} 								% PSTricks with the standard color package

\newtheorem{theorem}{{\bf Theorem}}
\newtheorem{property}{{\bf Property}}
\newtheorem{proposition}{{\bf Proposition}}
\newtheorem{corollary}[proposition]{{\bf Corollary}}
\newtheorem{lemma}[proposition]{{\bf Lemma}}

\AtBeginDocument{\renewcommand*\contentsname{Mục lục}}
\AtBeginDocument{\renewcommand*\refname{Tài liệu tham khảo}}
%\usepackage{fancyhdr}
\setlength{\headheight}{40pt}

\pagestyle{fancy}
\fancyhead{} % clear all header fields
\fancyhead[L]{
 \begin{tabular}{rl}
    \begin{picture}(25,15)(0,0)
    \put(0,-8){\includegraphics[width=8mm, height=8mm]{hcmut.png}}
    %\put(0,-8){\epsfig{width=10mm,figure=hcmut.eps}}
   \end{picture}&
	%\includegraphics[width=8mm, height=8mm]{hcmut.png} & %
	\begin{tabular}{l}
		\textbf{\bf \ttfamily Đại học Quốc gia Thành phố Hồ Chí Minh}\\
	\textbf{\bf \ttfamily Trường Đại học Bách khoa}
	\end{tabular} 	
 \end{tabular}
}
\fancyhead[R]{
	\begin{tabular}{l}
		\tiny \bf \\
		\tiny \bf 
	\end{tabular}  }
\fancyfoot{} % clear all footer fields
\fancyfoot[L]{\scriptsize \ttfamily Bài tập lớn môn Lập trình web (CO3049) - HK251}
\fancyfoot[R]{\scriptsize \ttfamily Trang {\thepage}/\pageref{LastPage}}
\renewcommand{\headrulewidth}{0.3pt}
\renewcommand{\footrulewidth}{0.3pt}


%%%
\setcounter{secnumdepth}{4}
\setcounter{tocdepth}{3}
\makeatletter
\newcounter {subsubsubsection}[subsubsection]
\renewcommand\thesubsubsubsection{\thesubsubsection .\@alph\c@subsubsubsection}
\newcommand\subsubsubsection{\@startsection{subsubsubsection}{4}{\z@}%
                                     {-3.25ex\@plus -1ex \@minus -.2ex}%
                                     {1.5ex \@plus .2ex}%
                                     {\normalfont\normalsize\bfseries}}
\newcommand*\l@subsubsubsection{\@dottedtocline{3}{10.0em}{4.1em}}
\newcommand*{\subsubsubsectionmark}[1]{}
\makeatother
 \usepackage{indentfirst}
 
% For code
\usepackage{color} % tô màu cho code
\usepackage{array}
\usepackage{tabularx, caption}
\usepackage{multirow}
\usepackage{multicol}
\usepackage{rotating}
\usepackage{graphics}
\usepackage{geometry}
\usepackage{setspace}
\usepackage{epsfig}
\usepackage{tikz}
\usetikzlibrary{arrows,snakes,backgrounds}
\usetikzlibrary{arrows,decorations.pathmorphing,backgrounds}
\hypersetup{urlcolor=blue,linkcolor=black,citecolor=black,colorlinks=true} 
\usepackage{subcaption}
\renewcommand\thesubfigure{\thefigure.\arabic{subfigure}}
\captionsetup[subfigure]{labelformat=simple, labelsep=space}
\usepackage{pstcol} % PSTricks with the standard color package
\usepackage{listings}
\definecolor{dkgreen}{rgb}{0,0.6,0}
\definecolor{gray}{rgb}{0.5,0.5,0.5}
\definecolor{mauve}{rgb}{0.58,0,0.82}
\definecolor{dkgreen}{rgb}{0,0.6,0}
\definecolor{gray}{rgb}{0.5,0.5,0.5}
\definecolor{mauve}{rgb}{0.58,0,0.82}
\lstset{frame=tb,
  language=C++,
  aboveskip=3mm,
  belowskip=3mm,
  showstringspaces=false,
  columns=flexible,
  basicstyle={\small\ttfamily},
  numbers=none,
  numberstyle=\tiny\color{gray},
  keywordstyle=\color{blue},
  commentstyle=\color{dkgreen},
  stringstyle=\color{mauve},
  breaklines=true,
  breakatwhitespace=true,
  tabsize=3
}

% Blue Frame
\usepackage{mdframed}
\usepackage{lipsum}
\usepackage{tcolorbox}

\usepackage{hyperref}

% -------------------------BEGIN DOCUMENT-------------------------------------
\begin{document}

\begin{titlepage}
\begin{center}
ĐẠI HỌC QUỐC GIA THÀNH PHỐ HỒ CHÍ MINH\\
TRƯỜNG ĐẠI HỌC BÁCH KHOA\\
KHOA KHOA HỌC VÀ KỸ THUẬT MÁY TÍNH
\end{center}

\vspace{1cm}

\begin{figure}[h!]
\begin{center}
\includegraphics[width=3cm]{hcmut.png}
\end{center}
\end{figure}

\vspace{1cm}


\begin{center}
\begin{tabular}{c}
\multicolumn{1}{l}{\textbf{{\Large LẬP TRÌNH WEB (CO3049)}}}\\
~~\\
\hline
\\
\multicolumn{1}{l}{\textbf{{\Large Bài tập lớn}}}\\
\\
\textbf{{\LARGE XÂY DỰNG TRANG WEB BÁN HÀNG}}\\
\textbf{{\LARGE  GIÀY DÉP}}\\
\\
\hline
\end{tabular}
\end{center}
\vspace{1cm}
\begin{center}
\textbf{\large{Lớp L03}}\\
\end{center}
\vspace{0,6cm}

\begin{table}[h]
\begin{tabular}{rrl}
\hspace{3 cm}
& \textbf{Giảng viên hướng dẫn:} & NGUYỄN HỮU HIẾU\\
& \textbf{Sinh viên:} & Trần Nhật Huy - 2252266 \\
&&  Lê Hoàng Việt- 2252903\\
&&  Hà Thái Toàn - 2213524\\
&&  Nguyễn Bảo Trâm - 2213572\\
& \textbf{Email liên hệ:} & {viet.lekhmtk22@hcmut.edu.vn}
\end{tabular}
\end{table}
\vspace{0,6cm}
\begin{center}
{TP. Hồ Chí Minh, ngày 08 tháng 12 năm 2025}
\end{center}
\end{titlepage}
\newpage
%%%%%%%%%%%%%%%%%%%%%%%%%%%%%%%%%
\newpage
\tableofcontents
\newpage
%--------------CONTENT-----------------------
\section{Giới thiệu đề tài}
\subsection{Tổng quan đề tài}
Trong bối cảnh chuyển đổi số mạnh mẽ diễn ra trên toàn cầu, thương mại điện tử (TMĐT) đã và đang trở thành xu hướng tất yếu trong hoạt động kinh doanh, đặc biệt tại Việt Nam – thị trường có quy mô TMĐT đạt 16,4 tỷ USD vào năm 2022 và dự kiến tăng trưởng khoảng 20\%\ vào năm 2023. Trong đó, mô hình B2C (Business-to-Consumer) ngày càng khẳng định vai trò quan trọng nhờ khả năng kết nối trực tiếp giữa doanh nghiệp và người tiêu dùng, giúp giảm thiểu chi phí trung gian, đồng thời mở rộng khả năng tiếp cận khách hàng không bị giới hạn bởi không gian và thời gian. Lĩnh vực thời trang giày dép là một trong những phân khúc đầy tiềm năng khi Việt Nam sở hữu dân số trẻ với hơn 60\%\ người dưới 35 tuổi (theo Tổng cục Thống kê, 2023), kéo theo nhu cầu lớn về các sản phẩm giày dép thời trang, đa dạng mẫu mã và chất lượng. Tuy nhiên, người tiêu dùng vẫn gặp nhiều khó khăn trong việc tìm kiếm sản phẩm chính hãng, so sánh giá cả hay cập nhật các xu hướng mới nhất.\\

Xuất phát từ thực tiễn này, website B2C bán giày được xây dựng nhằm giải quyết các vấn đề trên bằng cách cung cấp một nền tảng mua sắm trực tuyến tiện lợi, minh bạch và chuyên nghiệp. Website không chỉ giúp khách hàng dễ dàng tìm kiếm, lọc sản phẩm theo nhiều tiêu chí khác nhau, mà còn hỗ trợ đặt hàng nhanh chóng, theo dõi đơn hàng theo thời gian thực, cũng như nhận được các gợi ý phù hợp dựa trên hành vi mua sắm cá nhân. Về phía doanh nghiệp, hệ thống giúp tối ưu hóa quản lý kho hàng, chăm sóc khách hàng và phân tích dữ liệu để đưa ra chiến lược kinh doanh hiệu quả. Dự án tập trung phát triển website với các tính năng chính như: hệ thống quản lý sản phẩm đa dạng, tích hợp thanh toán trực tuyến an toàn, hệ thống quản lý đơn hàng thông minh, giao diện người dùng thân thiện và responsive, hệ thống quản trị nội dung linh hoạt, cùng các công cụ marketing và phân tích dữ liệu. Việc triển khai dự án này không chỉ đáp ứng nhu cầu thiết thực của thị trường mà còn góp phần thúc đẩy quá trình chuyển đổi số trong lĩnh vực thương mại điện tử tại Việt Nam, mở ra cơ hội phát triển bền vững cho các doanh nghiệp trong ngành thời trang giày dép.\\
\subsection{Sự cần thiết đề tài}
\begin{enumerate}
    \item \textit{Xu hướng chuyển đổi số trong thương mại:}
    Thị trường giày dép Việt Nam tăng trưởng 15-20\% mỗi năm, kéo theo nhu cầu mua sắm trực tuyến tăng mạnh.
    Theo Nielsen (2023), 75\% người tiêu dùng Việt Nam ưu tiên mua sắm online, đặc biệt là giới trẻ - chiếm 70\% dân số.
    
    \item \textit{Thực trạng thị trường giày dép:}
    Hệ thống cửa hàng truyền thống phân tán, thiếu minh bạch về chất lượng và giá cả.
    Người tiêu dùng gặp khó khăn khi so sánh sản phẩm giữa các cửa hàng và tìm kiếm thông tin chính xác.
    Chi phí mặt bằng và vận hành cửa hàng vật lý cao làm hạn chế đa dạng hóa sản phẩm.
\end{enumerate}
\subsection{Lợi ích và tiềm năng phát triển}
\begin{enumerate}
    \item \textit{Đối với doanh nghiệp:}
    Giảm chi phí vận hành 35-45\% so với mô hình truyền thống.
    Mở rộng thị trường toàn quốc và tiềm năng vươn ra quốc tế.
    Thu thập và phân tích dữ liệu khách hàng theo thời gian thực.
    
    \item \textit{Đối với người tiêu dùng:}
    Tiết kiệm 50-70\% thời gian mua sắm so với phương thức truyền thống.
    Dễ dàng so sánh giữa 1000+ mẫu giày từ nhiều thương hiệu.
    Đảm bảo chất lượng chính hãng với chính sách bảo hành rõ ràng.
    
    \item \textit{Tiềm năng phát triển:}
    Thị trường giày dép Việt Nam dự kiến đạt 3.5 tỷ USD vào 2025 (Statista).
    Tỷ lệ người dùng smartphone đạt 82\% dân số, tạo điều kiện mua sắm di động.
    90\% khách hàng trẻ tuổi ưa chuộng thanh toán qua ví điện tử hoặc thẻ quốc tế.
\end{enumerate}

Tóm lại, trong bối cảnh chuyển đổi số 4.0, việc phát triển website bán giày trực tuyến không chỉ đáp ứng xu hướng thị trường mà còn là giải pháp chiến lược giúp doanh nghiệp tối ưu hóa hoạt động kinh doanh và nâng cao trải nghiệm khách hàng trong ngành thời trang - một trong những lĩnh vực có tốc độ tăng trưởng nhanh nhất hiện nay.
\section{Cơ sở lý thuyết}
\begin{figure}[h]
    \centering
    \includegraphics[width=\textwidth]{IMG/Html-Css-Js.jpg} 
    \caption{HTML-CSS-JS.}
    \label{fig:my_label}
\end{figure}
\subsection{HTML}
HTML là viết tắt của cụm từ HyperText Markup Language, được sử dụng để tạo nên trang web. Trên một trang web có thể chứa nhiều trang và mỗi trang là một tài liệu HTML. Cha đẻ của HTML là Tim Berners-Lee, cũng chính là người sinh ra World Wide Web và là chủ tịch của World Wide Web Consortium (W3C - tổ chức thiết lập ra các chuẩn trên môi trường internet).\\

HTML là ngôn ngữ đánh dấu siêu văn bản, nó có vai trò xây dựng cấu trúc siêu văn bản trên một website, hoặc khai báo các tập tin kỹ thuật số như video, hình ảnh, nhạc. Nhóm chúng em đã tìm hiểu HTML (tìm hiểu cách chạy một file HTML, các thẻ HTML trên trang chủ W3C) để ứng dụng vào việc xây dựng cấu trúc cho các trang web chính của một khách sạn như: trang chủ, trang shop, trang thông tin khách hàng, trang giỏ hàng,... Phần lớn HTML đóng vai trò đưa ra cấu trúc cho trang web trong đồ án và xây dựng nên các trang web tĩnh. Cùng với CSS và JavaScript, HTML tạo ra bộ ba nền tảng kỹ thuật cho các website.\\

Với HTML ta có thể:
\begin{itemize}
    \item Thêm tiêu đề, định dạng đoạn văn, ngắt dòng điều khiển.
    \item Tạo danh sách, nhấn mạnh văn bản, tạo ký tự đặc biệt, chèn hình ảnh, tạo liên kết.
    \item Xây dựng bảng, điều khiển một số kiểu mẫu.
\end{itemize} 

Nhìn chung, HTML là ngôn ngữ markup, dễ học, dễ hiểu, dễ áp dụng. Tuy nhiên, một website được viết bằng HTML rất đơn giản. Để gây hứng thú với người truy cập, website cần có sự hỗ trợ của CSS và JavaScript.\\

\subsection{CSS}
CSS là viết tắt của Cascading Style Sheets, nó là ngôn ngữ được sử dụng để tìm và định dạng lại các phần tử được tạo bởi HTML. Nói ngắn gọn hơn là ngôn ngữ tạo phong cách cho trang web. CSS được phát triển bởi W3C vào năm 1996. Với CSS chúng ta có thể:
\begin{itemize}
    \item Tạo phong cách và định kiểu cho những yếu tố được viết dưới dạng ngôn ngữ đánh dấu, như HTML.
    \item Nội dung trang web sẽ tách bạch hơn trong việc định dạng hiển thị, từ đó quá trình cập nhật sẽ dễ dàng hơn.
    \item Tiết kiệm công sức nhờ điều khiển định dạng của nhiều trang web.
    \item Phân biệt cách hiển thị của trang web với nội dung chính của trang bằng cách điều khiển bố cục, màu sắc và font chữ.
\end{itemize}

Có thể nói, CSS gần như tạo nên bộ mặt của một website.\\

\subsection{JavaScript}
JS (tên đầy đủ là JavaScript) có tác dụng giúp chuyển website từ trạng thái tĩnh sang động, tạo tương tác để cải thiện hiệu suất máy chủ và nâng cao trải nghiệm người dùng. Hiểu đơn giản, JavaScript là ngôn ngữ được sử dụng rộng rãi khi kết hợp với HTML/CSS để thiết kế web động. Bên cạnh HTML và CSS, JavaScript cũng là một ngôn ngữ lập trình web phổ biến, được sử dụng rộng rãi trong suốt 20 năm qua. Tính đến 2016, có đến 92\% trang web hiện nay đang sử dụng JavaScript. Sử dụng JavaScript, ta sẽ:
\begin{itemize}
    \item Dễ dàng bắt đầu với các bước nhỏ, với thư viện ảnh, bố cục có tính thay đổi ... nhờ sự linh hoạt của JavaScript.
    \item Tăng cường các hành vi và kiểm soát mặc định của trình duyệt.
    \item Thông qua JavaScript, ta có thể kiểm tra dữ liệu đầu vào, nhằm giảm bớt công việc kiểm tra thủ công như kiểm tra tính hợp lệ của thông tin khách hàng, kiểm tra thông tin sản phẩm.
    \item JavaScript khá linh hoạt và có thể sử dụng ở nhiều nền tảng, trình duyệt và có thể được biên dịch bởi HTML trình duyệt web. Ta có thể truy cập và tương tác với website hiệu quả hơn, nhờ đặc tính gọn nhẹ mà chúng sẽ cho phép thực hiện các tác vụ trên trang web nhanh hơn.
\end{itemize}

Tóm lại: Tạo “sườn” web bằng HTML, làm cho trang web có nhiều màu sắc hơn bằng CSS, và tạo tính năng “động” cho trang web bằng JavaScript.

\subsection{PHP}
\begin{figure}[h]
    \centering
    \includegraphics[width=\textwidth]{IMG/php-mysql.jpg} 
    \caption{PHP và MySQL.}
    \label{fig:my_label}
\end{figure}

Với một ngôn ngữ làm web phổ biến như PHP, việc học tập nó là rất dễ dàng vì nguồn tài liệu tham khảo rất dồi dào trên Internet. Bản thân PHP có cú pháp và cấu trúc đơn giản, có thể nhanh chóng làm quen và sau khi làm quen, PHP có thể học thêm Framework Laravel.\\
\subsection*{Ưu điểm}
\begin{itemize}
    \item \textbf{Dễ học và sử dụng:} PHP là ngôn ngữ lập trình phổ biến với cú pháp đơn giản, phù hợp cho người mới bắt đầu.
    \item \textbf{Hiệu năng cao:} Được tối ưu cho phát triển web, PHP hoạt động nhanh và hiệu quả trên các ứng dụng server-side.
    \item \textbf{Hỗ trợ cộng đồng rộng lớn:} PHP có cộng đồng phát triển mạnh mẽ, cung cấp nhiều tài liệu, hướng dẫn và thư viện hỗ trợ.
    \item \textbf{Tương thích đa nền tảng:} PHP có thể chạy trên hầu hết các hệ điều hành, như Windows, Linux, và macOS.
    \item \textbf{Tích hợp dễ dàng với cơ sở dữ liệu:} PHP hỗ trợ tích hợp với các cơ sở dữ liệu phổ biến như MySQL, PostgreSQL, SQLite.
    \item \textbf{Miễn phí:} PHP là mã nguồn mở và không yêu cầu chi phí bản quyền.
\end{itemize}

\subsection*{Nhược điểm}
\begin{itemize}
    \item Còn hạn chế về cấu trúc của ngữ pháp. Nó không được thiết kế gọn gàng và không được đẹp mắt như những ngôn ngữ lập trình khác.
    \item PHP chỉ có thể hoạt động và sử dụng được trên các ứng dụng trong web. Đó chính là lý do khiến cho ngôn ngữ này khó có thể cạnh tranh được với những ngôn ngữ lập trình khác nếu như muốn phát triển và mở rộng hơn nữa trong lập trình.
\end{itemize}

\subsection{MySQL}
Về MySQL, là một trong những hệ thống quản trị cơ sở dữ liệu quan hệ SQL phổ biến nhất, phối hợp tốt với PHP. Nó được bảo trì bởi Oracle và được cập nhật thường xuyên. MySQL còn hỗ trợ indexing, bảo mật và cấp quyền ở một mức độ đơn giản.
\subsection*{Ưu điểm}
\begin{itemize}
    \item Về Giới hạn: Theo thiết kế, MySQL không có ý định làm tất cả và nó đi kèm với các hạn chế về chức năng mà một vài ứng dụng có thể cần.
    \item Về mức Độ tin cậy: Cách các chức năng cụ thể được xử lý với MySQL (ví dụ tài liệu tham khảo, các giao dịch, kiểm toán, ...) làm cho nó kém tin cậy hơn so với một số hệ quản trị cơ sở dữ liệu quan hệ khác.
    \item Ngoài ra, nếu số bản ghi của bạn lớn dần lên thì việc truy xuất dữ liệu của bạn là khá khó khăn. Khi đó, chúng ta sẽ phải áp dụng nhiều biện pháp để tăng tốc độ truy xuất dữ liệu như là chia tải database này ra nhiều server, hoặc tạo cache MySQL.
\end{itemize}

\subsection*{Nhược điểm của MySQL}
\begin{itemize}
    \item \textbf{Hạn chế về hiệu năng với dữ liệu lớn:} MySQL có thể không hoạt động hiệu quả khi xử lý cơ sở dữ liệu rất lớn hoặc ứng dụng có tải cao, so với các hệ quản trị như PostgreSQL hoặc Oracle.
    \item \textbf{Tính năng hạn chế:} Một số tính năng nâng cao, như hỗ trợ toàn diện cho các truy vấn phức tạp (CTE, window functions) hoặc giao dịch phân tán, chỉ mới được hỗ trợ trong các phiên bản gần đây hoặc còn hạn chế.
    \item \textbf{Thiếu sự tuân thủ hoàn toàn chuẩn SQL:} MySQL không tuân thủ chặt chẽ chuẩn SQL, có thể gây khó khăn khi di chuyển ứng dụng sang các hệ quản trị cơ sở dữ liệu khác.
    \item \textbf{Quản lý giao dịch hạn chế:} Mặc dù InnoDB hỗ trợ giao dịch, các công cụ lưu trữ khác trong MySQL, như MyISAM, không hỗ trợ giao dịch, gây khó khăn cho việc sử dụng nhất quán.
    \item \textbf{Tùy chọn sao lưu còn hạn chế:} Các phương pháp sao lưu mặc định của MySQL không mạnh mẽ bằng một số hệ thống khác, như Oracle hoặc PostgreSQL.
    \item \textbf{Không mạnh mẽ với phân tích dữ liệu:} MySQL không phải là lựa chọn tối ưu cho các ứng dụng phân tích dữ liệu phức tạp, vì thiếu các tính năng tối ưu cho mục đích này.
    \item \textbf{Cần cấu hình thủ công:} Hiệu suất của MySQL phụ thuộc nhiều vào việc cấu hình thủ công, điều này đòi hỏi kiến thức chuyên sâu từ quản trị viên.
    \item \textbf{Không tối ưu cho xử lý phân tán:} MySQL không được thiết kế để hỗ trợ tốt cho các hệ thống cơ sở dữ liệu phân tán, cần các công cụ bổ sung như Galera Cluster.
\end{itemize}

\subsection{Bootstrap}
\begin{center}
\includegraphics[width=\textwidth]{IMG/boostrap.png}\\
\small Bootstrap.
\end{center}
\textbf{Bootstrap} là một framework front-end mã nguồn mở được phát triển bởi Twitter. Nó được thiết kế để hỗ trợ xây dựng các giao diện web hiện đại và tối ưu hóa cho các thiết bị di động. Được giới thiệu lần đầu vào năm 2011, Bootstrap đã trở thành một trong những công cụ phổ biến nhất trong lĩnh vực thiết kế web nhờ tính tiện dụng và linh hoạt.

\textbf{Bootstrap} là một \textit{CSS framework} kết hợp HTML, CSS và JavaScript để cung cấp một bộ công cụ mạnh mẽ giúp lập trình viên và nhà thiết kế tạo giao diện người dùng (UI) nhanh chóng. Bootstrap hoạt động dựa trên triết lý thiết kế \textit{"mobile-first"}, nghĩa là ưu tiên tối ưu hóa cho thiết bị di động trước, sau đó mới mở rộng sang các màn hình lớn hơn.\\

\subsubsection*{Công dụng của Bootstrap}
Bootstrap cung cấp các thành phần và tính năng đa dạng, giúp phát triển giao diện web một cách nhanh chóng và dễ dàng. Một số công dụng chính gồm:

\begin{itemize}
    \item \textbf{Thiết kế giao diện đáp ứng (responsive):} Với hệ thống lưới (\textit{grid system}), Bootstrap cho phép tạo các giao diện hiển thị tốt trên mọi kích thước màn hình.
    \item \textbf{Thư viện thành phần UI phong phú:} Bao gồm các nút bấm, bảng biểu, form, menu, thanh điều hướng (navbar), modal, carousel, và nhiều thành phần khác.
    \item \textbf{Tích hợp sẵn các hiệu ứng JavaScript:} Bootstrap đi kèm với các plugin JavaScript như \textit{dropdowns}, \textit{modals}, \textit{tooltips}, \textit{popovers}, giúp giao diện trở nên sống động hơn.
    \item \textbf{Tùy chỉnh dễ dàng:} Người dùng có thể điều chỉnh các thành phần theo nhu cầu bằng cách ghi đè CSS hoặc sử dụng các biến Sass.
\end{itemize}

\subsubsection*{Ưu điểm của Bootstrap}
\begin{itemize}
    \item \textbf{Dễ sử dụng:} Không yêu cầu kiến thức chuyên sâu về CSS, bạn có thể dễ dàng tạo giao diện đẹp mắt chỉ bằng cách sử dụng các \textit{class} sẵn có.
    \item \textbf{Tiết kiệm thời gian:} Với các thành phần được xây dựng sẵn, Bootstrap giúp giảm đáng kể thời gian phát triển giao diện web.
    \item \textbf{Tính nhất quán:} Bootstrap đảm bảo rằng giao diện của bạn hiển thị nhất quán trên các trình duyệt khác nhau nhờ được kiểm thử rộng rãi.
    \item \textbf{Cộng đồng lớn và tài liệu đầy đủ:} Với tài liệu chi tiết và cộng đồng người dùng rộng lớn, bạn có thể dễ dàng tìm kiếm hướng dẫn, ví dụ và giải pháp khi gặp vấn đề.
    \item \textbf{Tương thích đa nền tảng:} Giao diện được xây dựng bằng Bootstrap tương thích tốt với mọi trình duyệt và thiết bị.
\end{itemize}

\subsubsection*{Hạn chế của Bootstrap}
\begin{itemize}
    \item \textbf{Giao diện tương đồng:} Các trang web sử dụng Bootstrap mặc định có thể dễ bị giống nhau nếu không tùy chỉnh sâu.
    \item \textbf{Hiệu suất:} Việc sử dụng nhiều \textit{class} và thành phần có thể làm tăng kích thước tệp CSS, dẫn đến hiệu suất tải trang bị ảnh hưởng.
    \item \textbf{Đường cong học tập:} Đối với người mới, việc hiểu và sử dụng đầy đủ các tính năng của Bootstrap có thể cần thời gian.
\end{itemize}

\textbf{Kết luận:} Bootstrap là một công cụ mạnh mẽ dành cho việc phát triển giao diện web hiện đại, đặc biệt là khi bạn muốn tối ưu hóa thời gian và chi phí. Với tính linh hoạt và khả năng tùy chỉnh, nó phù hợp cho cả những người mới bắt đầu lẫn các chuyên gia lập trình. Tuy nhiên, việc sử dụng hợp lý và tùy chỉnh theo nhu cầu riêng sẽ giúp bạn tận dụng tối đa các lợi ích mà Bootstrap mang lại.
\section{Thiết kế ứng dụng}

\subsection{Kiến trúc hệ thống}

Website bán giày trực tuyến được thiết kế dựa trên kiến trúc MVC (Model-View-Controller), đảm bảo sự tách biệt rõ ràng giữa dữ liệu, giao diện và logic xử lý. Cụ thể:

\begin{itemize}
    \item \textbf{Model}: Đại diện cho cấu trúc dữ liệu của ứng dụng, bao gồm các bảng trong cơ sở dữ liệu như Members, Products, Orders và các mối quan hệ giữa chúng.
    \item \textbf{View}: Hiển thị thông tin cho người dùng thông qua giao diện trực quan được phát triển bằng HTML5, CSS3 và JavaScript.
    \item \textbf{Controller}: Xử lý các yêu cầu từ người dùng, tương tác với Model và trả kết quả về View.
\end{itemize}

Nền tảng công nghệ bao gồm:
\begin{itemize}
    \item \textbf{Frontend}: HTML5, CSS3, JavaScript
    \item \textbf{Backend}: PHP
    \item \textbf{Database}: MySQL
    \item \textbf{Web Server}: Apache
\end{itemize}

Kiến trúc này cho phép phát triển ứng dụng có tính mở rộng cao, dễ bảo trì và đáp ứng các yêu cầu nghiệp vụ phức tạp.

\subsection{Các module chính của hệ thống}

\subsubsection{Module người dùng}
Module này quản lý mọi thông tin và hoạt động liên quan đến người dùng trên hệ thống:
\begin{itemize}
    \item \textbf{Đăng ký/Đăng nhập}: Cho phép người dùng tạo tài khoản mới hoặc đăng nhập vào hệ thống.
    \item \textbf{Quản lý thông tin cá nhân}: Người dùng có thể cập nhật thông tin cá nhân, địa chỉ giao hàng, thay đổi mật khẩu.
    \item \textbf{Lịch sử đơn hàng}: Hiển thị các đơn hàng đã đặt và trạng thái hiện tại.
\end{itemize}

\subsubsection{Module sản phẩm}
Module này quản lý toàn bộ danh mục và thông tin sản phẩm trong hệ thống:
\begin{itemize}
    \item \textbf{Chi tiết sản phẩm}: Hiển thị thông tin chi tiết về từng sản phẩm.
    \item \textbf{Quản lý kho hàng}: Cập nhật trạng thái.
    \item \textbf{Quản lý giá cả}: Thiết lập giá bán, giá khuyến mãi và các chính sách giá.
\end{itemize}

\subsubsection{Module giỏ hàng và thanh toán}
Module này xử lý quy trình mua hàng của người dùng:
\begin{itemize}
    \item \textbf{Thêm/Xóa sản phẩm}: Cho phép người dùng thêm hoặc xóa sản phẩm khỏi giỏ hàng.
    \item \textbf{Cập nhật số lượng}: Điều chỉnh số lượng sản phẩm trong giỏ hàng.
    \item \textbf{Tính toán giá}: Tự động tính toán tổng giá trị đơn hàng, bao gồm thuế và phí vận chuyển.
    \item \textbf{Tích hợp cổng thanh toán}: Kết nối với các cổng thanh toán trực tuyến.
    \item \textbf{Xác nhận đơn hàng}: Xử lý xác nhận và hoàn tất đơn hàng.
\end{itemize}

\subsubsection{Module quản trị}
Module này cung cấp công cụ cho quản trị viên điều hành hệ thống:
\begin{itemize}
    % \item \textbf{Quản lý người dùng}: Theo dõi và quản lý tài khoản người dùng.
    \item \textbf{Quản lý sản phẩm}: Thêm, sửa, xóa sản phẩm và danh mục.
    \item \textbf{Quản lý đơn hàng}: Xem và cập nhật trạng thái đơn hàng.
    \item \textbf{Quản lý bình luận}: Kiểm duyệt và quản lý các bình luận của người dùng.
    \item \textbf{Quản lý tin tức}: Thêm, sửa, xóa các bài tin tức.
\end{itemize}


\subsection{Giao diện người dùng}

\subsubsection{Trang chủ}
\begin{itemize}
    \item \textbf{Banner quảng cáo}: Hiển thị các chiến dịch tiếp thị và sản phẩm mới.
    \item \textbf{Sản phẩm nổi bật}: Trưng bày các sản phẩm bán chạy hoặc đang khuyến mãi.
    \item \textbf{Danh mục sản phẩm}: Liệt kê các danh mục sản phẩm chính.
    \item \textbf{Tin tức/Khuyến mãi}: Thông báo về các sự kiện và ưu đãi đặc biệt.
\end{itemize}

\subsubsection{Trang sản phẩm}
\begin{itemize}
    \item \textbf{Hiển thị sản phẩm dạng lưới}: Sắp xếp sản phẩm dạng lưới cho hiển thị tối ưu.
    \item \textbf{Phân trang}: Chia nhỏ danh sách sản phẩm thành nhiều trang.
\end{itemize}

\subsubsection{Trang chi tiết sản phẩm}
\begin{itemize}
    \item \textbf{Thông tin chi tiết}: Mô tả đầy đủ về sản phẩm, bao gồm kích thước, màu sắc.
\end{itemize}

\subsubsection{Trang Tin tức:}
\begin{itemize}
    \item \textbf{Danh sách bài viết}: Hiển thị các tin tức dưới dạng danh sách hoặc lưới.
    \item \textbf{Hình ảnh đại diện}: Mỗi tin tức có hình ảnh thumbnail minh họa.
    \item \textbf{Thông tin cơ bản}: Hiển thị tiêu đề, mô tả ngắn, đăng bởi ai và ngày đăng của tin tức.
\end{itemize}
\subsection{Quy trình xử lý đơn hàng}

Quy trình xử lý đơn hàng trên hệ thống bao gồm các bước sau:
\begin{enumerate}
    \item \textbf{Thêm vào giỏ hàng}: Người dùng chọn sản phẩm và thêm vào giỏ hàng.
    \item \textbf{Xác nhận thông tin}: Người dùng kiểm tra và cập nhật thông tin giao hàng.
    \item \textbf{Chọn phương thức thanh toán}: Người dùng chọn phương thức thanh toán phù hợp.
    \item \textbf{Xác nhận đơn hàng}: Người dùng xác nhận chi tiết đơn hàng.
    \item \textbf{Xử lý thanh toán}: Hệ thống xử lý giao dịch thanh toán.
    \item \textbf{Cập nhật trạng thái}: Hệ thống cập nhật trạng thái đơn hàng.
\end{enumerate}
\subsection{Cơ sở dữ liệu}

Hệ thống sử dụng cơ sở dữ liệu MySQL với các bảng chính sau:

\begin{figure}[h]
    \centering
    \includegraphics[width=\textwidth]{IMG/showimages/erd.png} 
    \caption{ERD minh họa cơ sở dữ liệu}
    \label{fig:erd_database}
\end{figure}

\section{Giới thiệu trang web}
\subsection{Về phía khách hàng}
        \begin{figure}[!htbp]
            \centering
            \includegraphics[width=\textwidth]{IMG/showimages/home.png} 
            \caption{Trang chủ.}
            \label{fig:homepage}
        \end{figure}
        \begin{figure}[!htbp]
            \centering
            \includegraphics[width=\textwidth]{IMG/showimages/products.png} 
            \caption{Trang sản phẩm.}
            \label{fig:products_page}
        \end{figure}
        \begin{figure}[!htbp]
            \centering
            \includegraphics[width=\textwidth]{IMG/showimages/product-detail.png} 
            \caption{Sản phẩm áp mã khuyến mãi.}
            \label{fig:discount_products}
        \end{figure}
        \begin{figure}[!h]
            \centering
            \begin{subfigure}{\textwidth}
                \centering
                \includegraphics[width=\textwidth]{IMG/showimages/news-list-1.png}
                \caption{Trang danh sách tin tức 1.}
                \label{fig:news_page_1}
            \end{subfigure}
             
            \begin{subfigure}{\textwidth}
                \centering
                \includegraphics[width=\textwidth]{IMG/showimages/news-list-2.png}
                \caption{Trang danh sách tin tức 2.}
                \label{fig:news_page_2}
            \end{subfigure}
        \end{figure}
        \begin{figure}[!htbp]
            \centering
            \includegraphics[width=\textwidth]{IMG/showimages/news-detail.png}
            \caption{Trang chi tiết tin tức.}
            \label{fig:news_page_3}
        \end{figure}
        \begin{figure}[!htbp]
            \centering
            \includegraphics[width=\textwidth]{IMG/showimages/cart.png}  
            \caption{Trang giỏ hàng.}
            \label{fig:cart_page}
        \end{figure}
        % \begin{figure}[!htbp]
        %     \centering
        %     \includegraphics[width=\textwidth]{IMG/showimages/update.jpg}  
        %     \caption{Trang chỉnh sửa thông tin cho khách hàng.}
        %     \label{fig:update_info}
        % \end{figure}
        \begin{figure}[!htbp]
            \centering
            \includegraphics[width=\textwidth]{IMG/showimages/login.png}  
            \caption{Trang đăng nhập.}
            \label{fig:login_page}
        \end{figure}
        \begin{figure}[!htbp]
            \centering
            \includegraphics[width=\textwidth]{IMG/showimages/register.png}  
            \caption{Trang đăng kí tài khoản.}
            \label{fig:register_page}
        \end{figure}
        \begin{figure}[!htbp]
            \centering
            \includegraphics[width=\textwidth]{IMG/showimages/about.png}  
            \caption{Trang giới thiệu.}
            \label{fig:about_page}
        \end{figure}
        \begin{figure}[!htbp]
            \centering
            \includegraphics[width=\textwidth]{IMG/showimages/qna.png}  
            \caption{Trang câu hỏi thường gặp.}
            \label{fig:qna_page}
        \end{figure}

\clearpage % Chèn ngắt trang rõ ràng trước khi bắt đầu subsection mới
\subsection{Về phía quản lý cửa hàng}
\begin{figure}[!htbp]
    \centering
    \includegraphics[width=\textwidth]{IMG/showimages/products-admin.png} 
    \caption{Sản Phẩm.}
    \label{fig:admin_products}
\end{figure}
\begin{figure}[!htbp]
    \centering
    \includegraphics[width=\textwidth]{IMG/showimages/products-admin-add.png} 
    \caption{Thêm sản phẩm.}
    \label{fig:add_product}
\end{figure}
\begin{figure}[!htbp]
    \centering
    \includegraphics[width=\textwidth]{IMG/showimages/products-admin-edit.png} 
    \caption{Sửa thông tin sản phẩm.}
    \label{fig:edit_product}
\end{figure}
\begin{figure}[!htbp]
    \centering
    \includegraphics[width=\textwidth]{IMG/showimages/orders-admin.png} 
    \caption{Quản lí đơn hàng.}
    \label{fig:admin_orders}
\end{figure}
% \begin{figure}[!htbp]
%     \centering
%     \includegraphics[width=\textwidth]{IMG/showimages/giao dien khach hang.png} 
%     \caption{Quản lí khách hàng.}
%     \label{fig:customer_management}
% \end{figure}
% \begin{figure}[!htbp]
%     \centering
%     \includegraphics[width=\textwidth]{IMG/showimages/thong tin.jpg} 
%     \caption{Thông tin khách hàng.}
%     \label{fig:customer_info}
% \end{figure}
\begin{figure}[!htbp]
    \centering
    \includegraphics[width=\textwidth]{IMG/showimages/news-admin.png} 
    \caption{Quản lý tin tức.}
    \label{fig:admin_news}
\end{figure}
\begin{figure}[!htbp]
    \centering
    \includegraphics[width=\textwidth]{IMG/showimages/news-admin-edit.png} 
    \caption{Sửa tin tức.}
    \label{fig:edit_news}
\end{figure}
\begin{figure}[!htbp]
    \centering
    \includegraphics[width=\textwidth]{IMG/showimages/news-admin-add.png} 
    \caption{Thêm thông tin tin tức mới.}
    \label{fig:add_news}
\end{figure}
\begin{figure}[!htbp]
    \centering
    \includegraphics[width=\textwidth]{IMG/showimages/comment-admin.png} 
    \caption{Quản lý bình luận.}
    \label{fig:admin_comment}
\end{figure}
% \begin{figure}[!htbp]
%     \centering
%     \includegraphics[width=\textwidth]{IMG/showimages/comment-admin.png} 
%     \caption{Quản lý mã khuyến mãi.}
%     \label{fig:admin_discount}
% \end{figure}
\begin{figure}[!htbp]
    \centering
    \includegraphics[width=\textwidth]{IMG/showimages/about_admin.png} 
    \caption{Thêm thông tin giới thiệu.}
    \label{fig:admin_about}
\end{figure}
\begin{figure}[!htbp]
    \centering
    \includegraphics[width=\textwidth]{IMG/showimages/qna_admin.png} 
    \caption{Thêm thông tin câu hỏi thường gặp.}
    \label{fig:admin_qna}
\end{figure}

\section{Hướng dẫn triển khai ứng dụng}

\subsection{Yêu cầu hệ thống}
Trước khi bắt đầu triển khai, đảm bảo hệ thống đã cài đặt đầy đủ các công cụ sau:

\begin{itemize}
  \item Docker Desktop (phiên bản 20.10 trở lên)
  \item Docker Compose (phiên bản 2.0 trở lên)
  \item Git (để clone source code từ repository)
\end{itemize}

\subsection{Quy trình triển khai}

\subsubsection{Bước 1: Cài đặt Docker Desktop}

Tải và cài đặt Docker Desktop phù hợp với hệ điều hành của bạn tại đường dẫn: 

\texttt{https://www.docker.com/products/docker-desktop}

Docker Desktop là một nền tảng tích hợp bao gồm Docker Engine, Docker CLI, Docker 

Compose và Kubernetes, giúp đơn giản hóa quá trình quản lý và vận hành containers trong 

môi trường development.

\subsubsection{Bước 2: Clone repository từ GitHub}

Mở terminal (Linux/macOS) hoặc Command Prompt/PowerShell (Windows) và thực hiện 

lệnh sau:

\begin{verbatim}
  git clone https://github.com/vietlecd/Ecommerce_Website
  cd Ecommerce_Website
\end{verbatim}

\begin{figure}[h!]
  \centering
  \includegraphics[width=0.9\textwidth]{IMG/showimages/git.png} 
  \caption{Repository GitHub chứa source code của dự án}
  \label{fig:github_repo}
\end{figure}

\subsubsection{Bước 3: Cấu hình và khởi động containers}

Di chuyển vào thư mục gốc của project và khởi động các services bằng Docker Compose:

\begin{verbatim}
  docker-compose up -d
\end{verbatim}

Tham số \texttt{-d} (detached mode) sẽ chạy containers ở background. Lệnh này thực hiện các tác 

vụ sau:

\begin{itemize}
  \item Build và khởi động containers cho Web Server (Apache/Nginx) và Database (MySQL)
  \item Cấu hình network cho phép các services giao tiếp với nhau
  \item Mount volumes để đồng bộ source code giữa host và container
  \item Expose ports để truy cập ứng dụng từ host machine
\end{itemize}

\subsubsection{Bước 4: Kiểm tra trạng thái deployment}

Xác nhận tất cả các containers đang chạy ổn định:

\begin{verbatim}
  docker-compose ps
\end{verbatim}

Output mong đợi sẽ hiển thị các services (\texttt{web}, \texttt{db}) với status là \texttt{Up} và port mappings tương 

ứng.

\subsubsection{Bước 5: Truy cập ứng dụng}

Mở trình duyệt web và truy cập địa chỉ sau để sử dụng ứng dụng:

\texttt{http://localhost:8080}

\subsection{Các lệnh quản lý thường dùng}

\begin{itemize}
  \item \textbf{Dừng toàn bộ services:} \texttt{docker-compose down}
  \item \textbf{Xem logs realtime:} \texttt{docker-compose logs -f [service\_name]}
  \item \textbf{Restart services:} \texttt{docker-compose restart [service\_name]}
  \item \textbf{Rebuild containers:} \texttt{docker-compose up -d --build}
  \item \textbf{Import database:} \texttt{docker-compose exec db mysql -u root -p shoesstore < database.sql}
  \item \textbf{Xem resource usage:} \texttt{docker stats}
\end{itemize}

\section{Nhiệm Vụ Từng Thành Viên}
\textbf{\Large Danh sách thành viên và mức độ đóng góp}

\begin{center}
\centering
\resizebox{\textwidth}{!}{%
    \begin{tabular}{|c|c|c|c|c|}
        \hline
\textbf{STT} & \textbf{Họ và tên} & \textbf{MSSV} & \textbf{Nhiệm vụ} & \textbf{Mức độ đóng góp}\\
\hline
%%%%%Student 1%%%%%%%%%%

1 & Trần Nhật Huy & 2252266 & Task 1 &  100\%\\
\hline 
2 & Hà Thái Toàn & 2213524 & Task 2 &  100\%\\
\hline 
3 & Lê Hoàng Việt & 2252903 & Task 3 &  100\%\\
\hline 
4 & Nguyễn Bảo Trâm & 2252162 & Task 4 &  100\%\\
\hline
    \end{tabular}%
    }
\end{center}

\section{Source code BTL}
\textbf{Link github}: \url{https://github.com/vietlecd/Ecommerce_Website}
\section{Tài liệu tham khảo}
\begin{enumerate}
    \item {"Hướng dẫn lập trình web từ A đến Z cho người mới bắt đầu".\\Tham khảo từ \textit{https://afterschool.fpt.edu.vn/lap-trinh-web-cho-nguoi-moi-bat-dau/}}
    \item {"5 Phút Tìm Hiểu HTML, CSS, Javascript Là Gì?". Tham khảo từ \textit{https://rikkei.edu.vn/html-css-javascript-la-gi/}}
    \item {"Javascript (JS) là gì?". Tham khảo từ \textit{https://aws.amazon.com/vi/what-is/javascript/}}
    \item {"MySQL là gì?". Tham khảo từ \textit{https://www.mcivietnam.com/blog-detail/mysql-la-gi/}}
    \item {"Bootstrap là gì? Cài đặt Bootstrap, web chuẩn responsive".\\Tham khảo từ \textit{https://wiki.matbao.net/bootstrap-la-gi-cai-dat-bootstrap-web-chuan-responsive/}}
\end{enumerate}
\end{document}

